\documentclass[12pt,a4paper]{article}
\usepackage[left=2.5cm,right=2.5cm,top=2.5cm,bottom=2.5cm]{geometry}
\usepackage[utf8]{inputenc}
\usepackage{amssymb, amsmath, amsthm}
\usepackage{graphics, graphicx}
\usepackage{hyperref}
\pagestyle{empty}
\hypersetup{
  colorlinks   = true, %Colours links instead of ugly boxes
  urlcolor     = blue, %Colour for external hyperlinks
}

\begin{document}
\textbf{Chapter 3 solutions  \hfill Hanna Gábor}

\begin{enumerate}
  \item
    \textit{Devise three example tasks of your own that fit into the MDP framework,
    identifying for each its states, actions, and rewards. Make the three examples as different
    from each other as possible. The framework is abstract and flexible and can be applied in
    many different ways. Stretch its limits in some way in at least one of your examples.}
    \begin{itemize}
      \item Games are an easy example. For example, in Snake the state is the current board, actions are
      turning left or right or continuing forward. Rewards are whether you picked up
      a chip or not.
      \item Public transportation development. Suppose we measured when and how many people
      use which line. (E.g. if we have an electronic ticket system, then this can be easily done.)
      Also suppose that we can simulate how people choose transportation vehicles. The actions
      would be like building a bus line here or a tram line there. The state is the graph created from
      the current lines and the possible new lines with their costs. Your current budget is
      also part of the state.
      The reward can be the number of people using public transport (the more the better)
      or the overall time needed for transportation (the less the better).
      We can add many things to this example. For example, you might need to bar some
      roads while you're building a new railway line.
    \end{itemize}
  \item
    \textit{Is the MDP framework adequate to usefully represent all goal-directed
    learning tasks? Can you think of any clear exceptions?}

    I'm not sure I'm working with the correct definition of "goal-directed" learning tasks
    in this exercise.

    An exception is if we can't make good intermediate rewards and reaching a good end-state
    is extremely unlikely. For example, if I don't know anything about how a car works
    and want a program that can learn to make a car. I can give a robot tools and give a reward
    if it builds a working car, but it will never get into a state like that.

    Another exception is if we can't really simulate the thing and can't try
    different methods in real life. E.g. I would like to make people happier. Should
    I start researching positive psychology or teach programming to poor kids or do something
    else? It's impossible to try taking different routes, to measure the resulting happiness
    or make a useful simulation for a virtual learning algorithm.
  \item
    \textit{Consider the problem of driving. You could define the actions in terms of
    the accelerator, steering wheel, and brake, that is, where your body meets the machine.
    Or you could define them farther out—say, where the rubber meets the road, considering
    your actions to be tire torques. Or you could define them farther in—say, where your
    brain meets your body, the actions being muscle twitches to control your limbs. Or you
    could go to a really high level and say that your actions are your choices of where to drive.
    What is the right level, the right place to draw the line between agent and environment?
    On what basis is one location of the line to be preferred over another? Is there any
    fundamental reason for preferring one location over another, or is it a free choice?}

    If it's about learning to drive a car, then the actions should be defined in terms
    of the accelerator, steering wheel and brake. If it's about learning to plan my
    days more efficiently then the actions should be defined as where I drive.

    I'm pretty sure that these are the good boundaries, but can't think of good
    rules.

  \item
    \textit{Give a table analogous to that in Example 3.3, but for $p(s', r|s, a)$. It
      should have columns for $s, a, s', r$, and $p(s', r|s, a)$, and a row for every 4-tuple
      for which $p(s', r | s, a) > 0$.}

    \begin{center}
      \begin{tabular}{ c|c|c|c|c }
        $s$ & $a$ & $s'$ & $r$ & $p(s', r | s, a)$ \\
       \hline
       high & search & high & $r_{search}$ & $\alpha$ \\
       high & search & low & $r_{search}$ & $1 - \alpha$ \\
       low & search & high & $-3$ & $1 - \beta$ \\
       low & search & low & $r_{search}$ & $\beta$ \\
       high & wait & high & $r_{wait}$ & $1$ \\
       low & wait & low & $r_{wait}$ & $1$ \\
       low & recharge & low & $0$ & $1$ \\
      \end{tabular}
    \end{center}

  \item
    \textit{The equations in Section 3.1 are for the continuing case and need to be
    modified (very slightly) to apply to episodic tasks. Show that you know the modifications
    needed by giving the modified version of (3.3).}

    \[
    \sum\limits_{s' \in \mathcal{S}^+} \sum\limits_{r \in \mathcal{R}} p(s', r | s, a) = 1 \text{, for all }
    s \in \mathcal{S}, a \in \mathcal{A}(s)
    \]

  \item
    \textit{Suppose you treated pole-balancing as an episodic task but also used
    discounting, with all rewards zero except for -1 upon failure. What then would the
    return be at each time? How does this return differ from that in the discounted, continuing
    formulation of this task?}

  The return would be $-\gamma^K$ where $K$ is the number of time steps before the next failure.
  Opposed to the continuing formulation, only the reward for the next failure is included
  in the return. In the continuing formulation, all the later failures have an effect on the
  return. (Although the later the failure occurs, the smaller the effect becomes.)

\item
  \textit{Imagine that you are designing a robot to run a maze. You decide to give it a
  reward of +1 for escaping from the maze and a reward of zero at all other times. The task
  seems to break down naturally into episodes—the successive runs through the maze—so
  you decide to treat it as an episodic task, where the goal is to maximize expected total
  reward (3.7). After running the learning agent for a while, you find that it is showing
  no improvement in escaping from the maze. What is going wrong? Have you effectively
  communicated to the agent what you want it to achieve?}

  The expected total reward is always $1$, the agent has no motivation to escape the maze
  early.

\item
  \textit{Suppose $\gamma = 0.5$ and the following sequence of rewards is received $R_1 = -1,$\\
  $R_2 = 2, R_3 = 6, R_4 = 3$, and $R_5 = 2$, with T = 5. What are $G_0, G_1, ..., G_5$? Hint:
  Work backwards.}

  \begin{align*}
    G_5 = 0\\
    G_4 = 2\\
    G_3 = 3 + 0.5 \cdot 2 = 4\\
    G_2 = 6 + 0.5 \cdot 4 = 8\\
    G_1 = 2 + 0.5 \cdot 8 = 6\\
    G_0 = -1 + 0.5 \cdot 6 = 2
  \end{align*}

\item
  \textit{Suppose $\gamma = 0.9$ and the reward sequence is $R_1 = 2$ followed by an infinite
  sequence of 7s. What are $G_1$ and $G_0$?}

  $G_1 = 7 + \gamma 7 + \gamma^2 7 + \dots = 7 \sum\limits_{i = 0}^{\infty} \gamma^i =
  7 \frac{1}{1 - \gamma} = \frac{7}{0.1} = 70$

  $G_0 = 2 + 0.9 \cdot 70 = 65$

\item
  \textit{Prove the second equality in (3.10).}
  \begin{align*}
    S_n &= \sum\limits_{i = 0}^n \gamma^i\\
    (\gamma - 1)S_n &= \gamma^{n + 1} - 1\\
    S_n &= \frac{\gamma^{n + 1} - 1}{\gamma - 1}\\
    \lim_{n \to \infty} S_n &= \frac{ - 1}{\gamma - 1} = \frac{1}{1 - \gamma}
  \end{align*}

\item
  \textit{If the current state is $S_t$, and actions are selected according to stochastic
  policy $\pi$, then what is the expectation of $R_{t+1}$ in terms of $\pi$ and the four-argument-function p (3.2)?}

  \[\mathbb{E}(R_{t + 1}) = \sum\limits_{a \in \mathcal{A}} \pi(a | S_t)
  \sum\limits_{r \in \mathcal{R}} r \sum\limits_{s' \in \mathcal{S}}p(s', r | s, a) \]

\item
  \textit{Give an equation for $v_{\pi}$ in terms of $q_{\pi}$ and $\pi$.}

  \[v_\pi(s) = \sum\limits_{a \in \mathcal{A}} \pi(a) q_\pi(s, a)\]

\item
  \textit{Give an equation for $q_{\pi}$ in terms of $v_{\pi}$ and the four-argument p.}

  \[
  q_\pi(s,a) = \sum\limits_{r \in \mathcal{R}} \sum\limits_{s' \in \mathcal{S}}p(s', r |s, a) (r + \gamma v_\pi(s'))
  \]

\item
  \textit{The Bellman equation (3.14) must hold for each state for the value function
  $v_\pi$ shown in Figure 3.2 (right) of Example 3.5. Show numerically that this equation holds
  for the center state, valued at +0.7, with respect to its four neighboring states, valued at
  +2.3, +0.4, -0.4, and +0.7. (These numbers are accurate only to one decimal place.)}

  Let $s$ denote the state with value $0.7$.

  \[v_\pi(s) = \sum\limits_{s' \text{is a neigbor of } s} 0.25 \cdot  (0 + 0.9 \cdot v_\pi(s')) =
  0.25 \cdot 0.9(2.3 + 0.4 - 0.4 + 0.7) = 0.25 \cdot 0.9 \cdot 2.6 \approx 0.7\]

\item
  \textit{In the gridworld example, rewards are positive for goals, negative for
  running into the edge of the world, and zero the rest of the time. Are the signs of these
  rewards important, or only the intervals between them? Prove, using (3.8), that adding a
  constant c to all the rewards adds a constant, $v_c$, to the values of all states, and thus
  does not affect the relative values of any states under any policies. What is $v_c$ in terms
  of c and $\gamma$?}

  Only the intervals are important. Adding a constant $c$ to all rewards results in
  the following change.

  \[
  v_\pi(s) = \mathbb{E}_\pi \Big(\sum\limits_{k = 0}^\infty \gamma^k (R_{t + k + 1} + c) \Bigm\lvert S_t = s\Big)
  = \mathbb{E}_\pi \Big(\sum\limits_{k = 0}^\infty \gamma^k R_{t + k + 1} \Bigm\lvert S_t = s\Big) +
  \sum\limits_{k = 0}^\infty \gamma^k c
  \]

  Hence, $v_c = \sum\limits_{k = 0}^\infty \gamma^k c$.

\item
  \textit{Now consider adding a constant c to all the rewards in an episodic task,
  such as maze running. Would this have any effect, or would it leave the task unchanged,
  as in the continuing task above? Why or why not? Give an example.}

  In the episodic case, adding a constant to all the rewards does have an effect,
  because the rewards after the end of the episode are still zeros. E.g. having a constant
  $1$ reward while you are in the maze, encourages staying in the maze, (After you are
  out of the maze, you won't collect rewards any more.) If the rewards are $-1$s, then
  you're encouraged to leave the maze. (You won't get penalties after you've find
  your way out of the maze.)

\item
  \textit{What is the Bellman equation for action values, that is, for $q_\pi$?
  It must give the action value $q_\pi(s, a)$ in terms of the action values, $q_\pi(s', a')$,
  of possible successors to the state–action pair (s, a).\\
  Hint: The backup diagram to the right corresponds to this equation.
  Show the sequence of equations analogous to (3.14), but for action
  values.}

  \begin{align*}
  q_\pi(s, a) & = \mathbb{E}(G_t | S_t = s, A_t = a)\\
  & = \mathbb{E}(R_{t + 1} + \gamma G_{t + 1} | S_t = s, A_t = a)\\
  & = \sum\limits_{s', r} p(s', r | s, a) (r + \gamma \mathbb{E} (G_{t + 1} | S_{t = 1} = s'))\\
  & = \sum\limits_{s', r} p(s', r | s, a) \Big(r + \gamma \sum\limits_{a'}\pi(a' | s') q_\pi(s', a')\Big)
  \end{align*}

\item
 \textit{The value of a state depends on the values of the actions possible in that
  state and on how likely each action is to be taken under the current policy. We can
  think of this in terms of a small backup diagram rooted at the state and considering each
  possible action. [Diagram.]
  Give the equation corresponding to this intuition and diagram for the value at the root
  node, $v_\pi(s)$, in terms of the value at the expected leaf node, $q_\pi(s, a)$, given
  $S_t = s$. This equation should include an expectation conditioned on following the policy,
  $\pi$. Then give a second equation in which the expected value is written out explicitly
  in terms of $\pi(a|s)$ such that no expected value notation appears in the equation.}

  \begin{align*}
    v_\pi(s) = \mathbb{E}(q_\pi(s, a) | S_t = s, a = A_t) = \sum\limits_a \pi(a|s) q_\pi(s, a)
  \end{align*}

\item
  \textit{The value of an action, $q_\pi(s, a)$, depends on the expected next reward and
  the expected sum of the remaining rewards. Again we can think of this in terms of a
  small backup diagram, this one rooted at an action (state–action pair) and branching to
  the possible next states: [Diagram.]
  Give the equation corresponding to this intuition and diagram for the action value,
  $q\pi(s, a)$, in terms of the expected next reward, $R_{t+1}$, and the expected next state value,
  $v\pi(S_{t+1})$, given that $S_t = s$ and $A_t = a$. This equation should include an expectation but
  not one conditioned on following the policy. Then give a second equation, writing out the
  expected value explicitly in terms of $p(s', r|s, a)$ defined by (3.2), such that no expected
  value notation appears in the equation.}

  \[q_\pi(s, a) = \mathbb{E}(R_{t + 1}) + \gamma \mathbb{E} (v_\pi (S_{t + 1}) | s = S_t, a = A_t)
  = \sum\limits_{s', r} p(s', r | s, a) (r + \gamma v_\pi(s'))\]

\item
  \textit{Draw or describe the optimal state-value function for the golf example.}

  \[
  v_\pi(s) =
  \begin{cases}
    -\infty & \text{if the ball is in the sand}\\
    -1 & \text{if the ball is in the green area}\\
    q_*(s, \text{driver}) & \text{otherwise}
  \end{cases}
  \]

\item
  \textit{Draw or describe the contours of the optimal action-value function for
  putting, $q_\pi(s, putter)$, for the golf example.}

  \[
  q_\pi(s, \text{putter}) =
  \begin{cases}
    -\infty & \text{if the ball is in the sand}\\
    -1 & \text{if the ball is in the green area}\\
    -2 & \text{if } v_{putt} = -2\\
    -3 & \text{if } v_{putt} < -2 \text{ and we can put the ball somewhere where }\\
    & q_*(s, \text{driver}) = -2\\
    -4 & \text{otherwise}\\
  \end{cases}
  \]

\item
  \textit{Consider the continuing MDP shown on to the right. [Diagram.] The only decision
  to be made is that in the top state, where two actions are available, left and right.
  The numbers show the rewards that are received deterministically after
  each action. There are exactly two deterministic policies,
  $\pi_{left}$ and $\pi_{right}$. What policy is optimal if $\gamma = 0$? If $\gamma = 0.9$?
  If $\gamma = 0.5$?}

  If $\gamma = 0$, then only the next reward matters, so the optimal policy is
  $\pi_{left}$.

  If $\gamma = 0.9$, then the optimal policy is $\pi_{right}$, because $0 + 0.9 \cdot 2 > 1 + 0.9 \cdot 0$.

  If $\gamma = 0.5$, then all the policies are optimal, because $0 + 0.5 \cdot 1 = 1 + 0.5 \cdot 0.$

\item
  \textit{Give the Bellman equation for $q_\ast$ for the recycling robot.}
  \begin{align*}
    q_\ast(h, s) &= p(h|h, s) (r(h, s, h) + \gamma \max(q_\ast(h, s), q_\ast(h, w))\\
    & + p(l|h, s) (r(h, s, l) + \gamma \max(q_\ast(l, s), q_\ast(l, w) + q_\ast(l, r))\\
    & = \alpha(r_{search} + \gamma \max(q_\ast(h, s), q_\ast(h, w))\\
    & + (1 - \alpha)(r_{search} + \gamma \max(q_\ast(l, s), q_\ast(l, w) + q_\ast(l, r))\\
    q_\ast(h, w) &= p(h|h, w) (r(h, w, h) + \gamma \max(q_\ast(h, s), q_\ast(h, w))\\
    & + p(l|h, w) (r(h, w, l) + \gamma \max(q_\ast(l, s), q_\ast(l, w) + q_\ast(l, r))\\
    & = r_{wait} + \gamma \max(q_\ast(h, s), q_\ast(h, w))\\
    q_\ast(l, s) &= p(h|l, s) (r(l, s, h) + \gamma \max(q_\ast(h, s), q_\ast(h, w))\\
    & + p(l|l, s) (r(l, s, l) + \gamma \max(q_\ast(l, s), q_\ast(l, w) + q_\ast(l, r))\\
    & = (1 - \beta)(-3 + \gamma \max(q_\ast(h, s), q_\ast(h, w))\\
    & + \beta(r_{search} + \gamma \max(q_\ast(l, s), q_\ast(l, w) + q_\ast(l, r))\\
    q_\ast(l, w) &= p(h|l, w) (r(l, w, h) + \gamma \max(q_\ast(h, s), q_\ast(h, w))\\
    & + p(l|l, w) (r(l, w, l) + \gamma \max(q_\ast(l, s), q_\ast(l, w) + q_\ast(l, r))\\
    & = r_{wait} + \gamma \max(q_\ast(l, s), q_\ast(l, w) + q_\ast(l, r))\\
    q_\ast(l, r) &= p(h|l, r) (r(l, r, h) + \gamma \max(q_\ast(h, s), q_\ast(h, w))\\
    & + p(l|l, r) (r(l, r, l) + \gamma \max(q_\ast(l, s), q_\ast(l, w) + q_\ast(l, r))\\
    & = \gamma \max(q_\ast(h, s), q_\ast(h, w))\\
  \end{align*}

\end{enumerate}
\end{document}
