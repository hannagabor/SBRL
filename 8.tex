\documentclass[12pt,a4paper]{article}
\usepackage[left=2.5cm,right=2.5cm,top=2.5cm,bottom=2.5cm]{geometry}
\usepackage[utf8]{inputenc}
\usepackage{amssymb, amsmath, amsthm}
\usepackage{hyperref}
\usepackage{algorithmic, algorithm}
\usepackage{graphics, graphicx}

\DeclareMathOperator*{\argmax}{argmax}
\pagestyle{empty}
\hypersetup{
  colorlinks   = true, %Colours links instead of ugly boxes
  urlcolor     = blue, %Colour for external hyperlinks
}

\begin{document}
\textbf{Chapter 8 solutions  \hfill Hanna Gábor}

\begin{enumerate}
  \item
    \textit{The nonplanning method looks particularly poor in Figure 8.3 because it is
    a one-step method; a method using multi-step bootstrapping would do better. Do you
    think one of the multi-step bootstrapping methods from Chapter 7 could do as well as
    the Dyna method? Explain why or why not.}

    I think it can come close. An $n$-step bootstrpping algorithm with big enough $n$ (or a
    Monte Carlo method) would update at the end of the first episode all the action-values we
    encountered during the episode.
    I expect the policy based on these updated action-values to be quite good.

  \item
    \textit{Why did the Dyna agent with exploration bonus, Dyna-Q+, perform
    better in the first phase as well as in the second phase of the blocking and shortcut
    experiments?}

    At first, both algorithms might find an okay policy that is suboptimal. Due
    to the exploration, Dyna-Q+ will realize it sooner that it's a suboptimal policy.

  \item
    \textit{Careful inspection of Figure 8.5 reveals that the difference between Dyna-Q+
    and Dyna-Q narrowed slightly over the first part of the experiment. What is the reason
    for this?}

    After finding the optimal path, Dyna-Q always uses that path, whereas Dyna-Q+
    does some exploration from time to time.
\end{enumerate}
\end{document}
