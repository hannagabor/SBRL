\documentclass[12pt,a4paper]{article}
\usepackage[left=2.5cm,right=2.5cm,top=2.5cm,bottom=2.5cm]{geometry}
\usepackage[utf8]{inputenc}
\usepackage{amssymb, amsmath, amsthm}
\usepackage{hyperref}
\usepackage{algorithmic, algorithm}
\usepackage{graphics, graphicx}
\DeclareMathOperator*{\argmax}{argmax}
\pagestyle{empty}
\hypersetup{
  colorlinks   = true, %Colours links instead of ugly boxes
  urlcolor     = blue, %Colour for external hyperlinks
}

\begin{document}
\textbf{Chapter 10 solutions  \hfill Hanna Gábor}

\begin{enumerate}
  \item \textit{We have not explicitly considered or given pseudocode for any Monte Carlo
  methods in this chapter. What would they be like? Why is it reasonable not to give
  pseudocode for them? How would they perform on the Mountain Car task?}

  The Monte Carlo is basically $n$-step Sarsa with $n = T$. I expect it to perform
  poorly on the Mountain Car example: until it doesn't reach the goal, it doesn't
  learn anything, so the first episode might be really long.
  As opposed to this, $n$-step Sarsa tries new actions after it sees
  that the previous actions didn't lead to the end of the episode, so it will
  get to the goal line eventually.
\end{enumerate}
\end{document}
