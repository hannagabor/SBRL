\documentclass[12pt,a4paper]{article}
\usepackage[left=2.5cm,right=2.5cm,top=2.5cm,bottom=2.5cm]{geometry}
\usepackage[utf8]{inputenc}
\usepackage{amssymb, amsmath, amsthm}
\usepackage{hyperref}
\usepackage{algorithmic, algorithm}
\usepackage{graphics, graphicx}
\DeclareMathOperator*{\argmax}{argmax}
\pagestyle{empty}
\hypersetup{
  colorlinks   = true, %Colours links instead of ugly boxes
  urlcolor     = blue, %Colour for external hyperlinks
}

\begin{document}
\textbf{Chapter 9 solutions  \hfill Hanna Gábor}

\begin{enumerate}
  \item \textit{Show that tabular methods such as presented in Part I of this book are a
    special case of linear function approximation. What would the feature vectors be?}
    
    The feature would be an indicator of the state: the feature vector's length would be
    equal to the number of states, and $x(s_i)$ would be a vector that contains $1$ at
    the $i^{th}$ coordinate and $0$ at the others. The weight vector would also have
    the same length as the number of states. The $i^th$ weight would be our value estimate
    for $s_i$. When we're doing the update, the gradient of $x(s_i)$ is $0$ everywhere except
    for coordinate $i$, where it's $1$, so we would only update the weight for the current state.
    
  \item \textit{Why does (9.17) define $(n + 1)^k$ distinct features for dimension $k$?}
    
    For each state $s_j$, we choose one of $n + 1$ options: $s_j^0, s_j^1 \dots, s_j^n$.
    There are $k$ states, we make a choice for each them: we can do that ($(n + 1)^k$)-many
    ways.


\end{enumerate}
\end{document}
