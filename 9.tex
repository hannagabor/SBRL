\documentclass[12pt,a4paper]{article}
\usepackage[left=2.5cm,right=2.5cm,top=2.5cm,bottom=2.5cm]{geometry}
\usepackage[utf8]{inputenc}
\usepackage{amssymb, amsmath, amsthm}
\usepackage{hyperref}
\usepackage{algorithmic, algorithm}
\usepackage{graphics, graphicx}
\DeclareMathOperator*{\argmax}{argmax}
\pagestyle{empty}
\hypersetup{
  colorlinks   = true, %Colours links instead of ugly boxes
  urlcolor     = blue, %Colour for external hyperlinks
}

\begin{document}
\textbf{Chapter 9 solutions  \hfill Hanna Gábor}

\begin{enumerate}
  \item \textit{Show that tabular methods such as presented in Part I of this book are a
    special case of linear function approximation. What would the feature vectors be?}
    
    The feature would be an indicator of the state: the feature vector's length would be
    equal to the number of states, and $x(s_i)$ would be a vector that contains $1$ at
    the $i^{th}$ coordinate and $0$ at the others. The weight vector would also have
    the same length as the number of states. The $i^th$ weight would be our value estimate
    for $s_i$. When we're doing the update, the gradient of $x(s_i)$ is $0$ everywhere except
    for coordinate $i$, where it's $1$, so we would only update the weight for the current state.
    
  \item \textit{Why does (9.17) define $(n + 1)^k$ distinct features for dimension $k$?}
    
    For each state $s_j$, we choose one of $n + 1$ options: $s_j^0, s_j^1 \dots, s_j^n$.
    There are $k$ states, we make a choice for each them: we can do that ($(n + 1)^k$)-many
    ways.

  \item \textit{What n and $c_{i,j}$ produce the feature vectors
    \[x(s) = (1, s_1, s_2, s_1s_2, s_1^2, s_2^2, s_1s_2^2, s_1^2s_2, s_1^2s_2^2)^T?\]}
    \vspace{-1cm}
    \begin{align*}
    n &= 2 \\
    c_{1, 1} &= c_{1, 2} = 0\\
    c_{2, 1} &= 1, c_{2, 2} = 0\\
    c_{3, 1} &= 0, c_{3, 2} = 1\\
    c_{4, 1} &= 1, c_{4, 2} = 1\\
    c_{5, 1} &= 2, c_{5, 2} = 0\\
    c_{6, 1} &= 0, c_{6, 2} = 2\\
    c_{7, 1} &= 1, c_{7, 2} = 2\\
    c_{8, 1} &= 2, c_{8, 2} = 1\\
    c_{9, 1} &= 2, c_{9, 2} = 2\\
    \end{align*}
  
  \item \textit{Suppose we believe that one of two state dimensions is more likely to have
  an effect on the value function than is the other, that generalization should be primarily
  across this dimension rather than along it. What kind of tilings could be used to take
  advantage of this prior knowledge?}

  Let's say it's the first dimension that is more likely to have an effect on the value
  function. In this case, the tiles should be short in the first dimension and long in
  the second dimension. This way, if we learn from an example $(s_1, s_2)$, we update
  many $(s_1, x)$ states and less $(x, s_2)$ states.

  \item \textit{Suppose you are using tile coding to transform a seven-dimensional continuous
  state space into binary feature vectors to estimate a state value function
  $\hat{v}(s,w) \approx v_\pi(s)$. You believe that the dimensions do not interact strongly,
  so you decide to use eight tilings of each dimension separately (stripe tilings), for 
  $7 \cdot 8 = 56$ tilings. In addition, in case there are some pairwise interactions between
  the dimensions, you also take all $\binom{7}{2} = 21$ pairs of dimensions and tile each
  pair conjunctively with rectangular tiles. You make two tilings for each pair of dimensions,
  making a grand total of $21 \cdot 2 + 56 = 98$ tilings. Given these feature vectors, you suspect
  that you still have to average out some noise, so you decide that you want learning to be
  gradual, taking about 10 presentations with the same feature vector before learning nears
  its asymptote. What step-size parameter $\alpha$ should you use? Why?}

  Each state will be contained in $98$ tiles, that means there will be $98$ ones in the feature vector.
  Based on the rule of thumb before the exercise,
  \[\alpha = \frac{1}{10 \mathbb{E}(x^Tx)} = \frac{1}{10 \cdot 98} = \frac{1}{980}\]
  should be a good choice.

  \item \textit{If $\tau = 1$, prove that (9.19) together with (9.7) results in the error being
  reduced to zero in one update.}

  The goal is to prove that after the update $\mathbb{E}(x_t^Tw_{t + 1}) = \mathbb{E}(U_t)$.

  \begin{align*}
    x_t^T w_{t + 1} &= x_t^T(w_t + \alpha(U_t - \hat{v}(S_t, w_t)) \nabla \hat{v}(S_t, w_t))\\
    &= x_t^T\Big(w_t + \frac{1}{\mathbb{E}(x^Tx)} (U_t - x_t^T w_t)\Big)x_t\\
    &= x_t^T w_t + U_t\frac{x_t^T x_t}{\mathbb{E}(x^Tx)} - x_t^T w_t \frac{x_t^T x_t}{\mathbb{E}(x^Tx)}\\
    \mathbb{E}(x_t^T w_{t + 1}) &= \mathbb{E}(x_t^T w_t)
    + \mathbb{E}\Big(U_t\frac{x_t^T x_t}{\mathbb{E}(x^Tx)}\Big)
    - \mathbb{E}\Big(x_t^T w_t \frac{x_t^T x_t}{\mathbb{E}(x^Tx)}\Big)
  \end{align*}

  If $x_t^T x_t$ is independent from $U_t$ and $x_t^Tw_t$, then $\mathbb{E}(x_t^Tw_{t + 1}) = \mathbb{E}(U_t)$
  follows from this. Otherwise it doesn't. I'm not sure if the statement is true if
  these are not independent.

  \item \textit{One of the simplest artificial neural networks consists of a single semi-linear
  unit with a logistic nonlinearity. The need to handle approximate value functions of this
  form is common in games that end with either a win or a loss, in which case the value of
  a state can be interpreted as the probability of winning. Derive the learning algorithm
  for this case, from (9.7), such that no gradient notation appears.}

  The derivative of the sigmoid function $\sigma$ is $\sigma'(x) = \sigma(x)
  (1 - \sigma(x)$.

  \begin{align*}
    w_{t + 1} &= w_t + \alpha\sigma(w_t^Tx_t) \nabla \sigma(w_t^Tx_t)\\
    &= w_t + \alpha \sigma(w_t^Tx_t) \sigma'(w_t^Tx_t) x_t \\
    &= w_t + \alpha \sigma(w_t^Tx_t)^2(1 - \sigma(w_t^Tx_t)) x_t \\
  \end{align*}

\end{enumerate}
\end{document}
