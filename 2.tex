\documentclass[12pt,a4paper]{article}
\usepackage[left=2.5cm,right=2.5cm,top=2.5cm,bottom=2.5cm]{geometry}
\usepackage[utf8]{inputenc}
\usepackage{amssymb, amsmath, amsthm}
\usepackage{graphics, graphicx}
\pagestyle{empty}
\newtheorem{lemma}{Lemma}
\newtheorem{thm}{Theorem}

\begin{document}
\textbf{Chapter 2 solutions  \hfill Hanna Gábor}

\begin{enumerate}
  \item
    \textit{In $\epsilon$-greedy action selection, for the case of two actions and $\epsilon = 0.5$, what is the probability that the greedy action is selected?}

    $\mathbb{P}(\text{greedy action is selected}) = 0.5 + 0.5/2 = 0.75.$

  \item
    \textit{Bandit example. Consider a $k$-armed bandit problem with $k = 4$ actions, denoted $1$, $2$, $3$, and $4$. Consider applying to this problem a bandit algorithm using $\epsilon$-greedy action selection, sample-average action-value estimates, and initial estimates of $Q_1(a) = 0$, for all $a$. Suppose the initial sequence of actions and rewards is $A_1 = 1$, $R_1 = 1$, $A_2 = 2$, $R_2 = 1$, $A_3 = 2$, $R_3 = 2$, $A_4 = 2$,
    $R_4 = 2$, $A_5 = 3$, $R_5 = 0$. On some of these time steps the $\epsilon$ case may have occurred, causing an action to be selected at random. On which time steps did this definitely occur? On which time steps could this possibly have occurred?}

    Any action can be an explorative move.

    What were the greedy options in different time steps?\\
    Step $1$: all actions have $0$ estimated values. Every action is a greedy choice.\\
    Step $2$: $Q_1(1) = 1$. The greedy choice now is $1$. $A_2 = 2$ must have been an\\ explorative move.\\
    Step $3$: $Q_2(2) = 1$. The greedy choice is either $1$ or $2$.\\
    Step $4$: $Q_3(2) = 1.5$. The greedy choice is $2$.\\
    Step $5$: $Q_4(2) = 1.67$. The greedy choice is $2$. $A_5 = 3$ must have been an explorative move.

    On time steps $2$ and $5$ a random action must have been selected.

\end{enumerate}
\end{document}
