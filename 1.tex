\documentclass[12pt,a4paper]{article}
\usepackage[left=2.5cm,right=2.5cm,top=2.5cm,bottom=2.5cm]{geometry}
\usepackage[utf8]{inputenc}
\usepackage{amssymb, amsmath, amsthm}
\usepackage{graphics, graphicx}
\pagestyle{empty}
\newtheorem{lemma}{Lemma}
\newtheorem{thm}{Theorem}

\begin{document}
\textbf{Chapter 1 solutions  \hfill Hanna Gábor}\\

\begin{enumerate}
  \item
    \textit{Self-Play. Suppose, instead of playing against a random opponent, the reinforcement learning algorithm described above played against itself, with both sides learning. What do you think would happen in this case? Would it learn a different policy for selecting moves?}

    Yes, I expect it to learn a different policy: an optimal policy against an optimal policy. (Apart from the exploratory moves.)

    This can be shown by induction. For each state that necessarily leads to the end of the game, the greedy action will be the one that leads to a victory (if there exists such an action) and thus the value of this state after enough time steps will be really close to $1$, if there exists a winning action and really close to $0$ otherwise. Suppose that the values of states leading to the end of the game in maximum $k$ steps are all really close to $0$ or $1$ reflecting who wins the game if both players play perfectly. For a state that leads to the end of the game in maximum $k + 1$ steps, all the actions lead to states leading to an end in maximum $k$ steps, hence after enough time steps, the greedy action in this state will be optimal and the value will be close to $0$ or $1$.

   \item
     \textit{Symmetries. Many tic-tac-toe positions appear different but are really the same because of symmetries. How might we amend the learning process described above to take advantage of this? In what ways would this change improve the learning process?}

     We could represent the states that are symmetricly equivalent as one state instead of many states. This would make the learning faster.

     \textit{Now think again. Suppose the opponent did not take advantage of symmetries. In that case, should we? Is it true, then, that symmetrically equivalent positions should necessarily have the same value?}

      No. If the opponent makes different choices in two symmetrically equivalent states, then we might have a dfferent winning chance in those, so we should not think of them as the same state.

  \item
    \textit{Greedy Play. Suppose the reinforcement learning player was greedy, that is, it always played the move that brought it to the position that it rated the best. Might it  learn to play better, or worse, than a nongreedy player? What problems might occur?}

    If the other player follows a deterministic strategy, then a greedy method would be better. If the other player does not play deterministically, then this is not the case. Suppose that from a state $S_0$ the agent has two actions that lead to states $S_1$ and $S_2$. It can happen that at the beginning the learning agent loses from state $S_1$ because of bad luck, so it assigns a lower value for $S_1$ than the true chance of winning. If $S_2$ has a higher value than this assigned value, then the agent will probably always choose $S_2$. In this case, if the agent has a lower chance of winning from $S_2$ than from $S_1$, the agent probably will not figure it out.

  \item
    \textit{Learning from Exploration. Suppose learning updates occurred after all moves, including exploratory moves. If the step-size parameter is appropriately reduced over time (but not the tendency to explore), then the state values would converge to a different set of probabilities. What (conceptually) are the two sets of probabilities computed when we do, and when we do not, learn from exploratory moves?}

    If we learn from exploratory moves, then we learn the probabilities of winning in case we follow the strategy that occasionally make exploratory moves. If we do not learn from exploratory moves, then we learn the probability of winning in case of the greedy strategy.

    \textit{Assuming that we do continue to make exploratory moves, which set of probabilities might be better to learn? Which would result in more wins?}

    If we continue to make exploratory moves, then we should learn from exploratory moves as well. Let's see an example. Suppose the agent can choose from two states with similar winning probabilities if the agent plays greedy. In the first state, the agent needs to do exactly the right moves after being in that state to get that chance of winning. In the second state, the agent can do whatever it wants, the winning probability will not change. It might happen that if the agent continues to make exploratory moves, it has a lower chance of winning from the first state than from the second state, but if it plays greedily, then it has a better chance if it chooses an action that leads to the first state.

  \item
    \textit{Other Improvements. Can you think of other ways to improve the reinforcement learning player? Can you think of any better way to solve the tic-tac-toe problem as posed?}

    If the opponent is deterministic, then we can go through all the states (starting from the leaves) and determine the best action from there.

    We could also use the result of a game to get better estimates not only for the last state, but for previous states as well. I am not sure if that would help overall, but it would help to get better estimates than $0.5$ sooner. 


\end{enumerate}
\end{document}
